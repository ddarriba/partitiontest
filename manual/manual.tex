\documentclass[11pt,twoside,a4paper]{article}
\usepackage{pslatex,palatino,avant,graphicx}
\usepackage[usenames,dvipsnames]{color}
\usepackage[margin=2cm]{geometry}
\usepackage{url}
\usepackage[square,sort]{natbib}

\usepackage{listings}
\lstset{ %
  backgroundcolor=\color{white},   % choose the background color; you must add \usepackage{color} or \usepackage{xcolor}
  basicstyle=\footnotesize,        % the size of the fonts that are used for the code
  breakatwhitespace=false,         % sets if automatic breaks should only happen at whitespace
  breaklines=true,                 % sets automatic line breaking
  captionpos=b,                    % sets the caption-position to bottom
  commentstyle=\color{LimeGreen},  % comment style
  % deletekeywords={...},            % if you want to delete keywords from the given language
  escapeinside={\%*}{*)},          % if you want to add LaTeX within your code
  extendedchars=true,              % lets you use non-ASCII characters; for 8-bits encodings only, does not work with UTF-8
  frame=single,                    % adds a frame around the code
  keepspaces=true,                 % keeps spaces in text, useful for keeping indentation of code (possibly needs columns=flexible)
  keywordstyle=\color{BlueGreen},  % keyword style
  % language=Java,                   % the language of the code
  % numbers=left,                    % where to put the line-numbers; possible values are (none, left, right)
  % numbersep=5pt,                   % how far the line-numbers are from the code
  % numberstyle=\tiny\color{Gray},   % the style that is used for the line-numbers
  rulecolor=\color{black},         % if not set, the frame-color may be changed on line-breaks within not-black text (e.g. comments (green here))
  showspaces=false,                % show spaces everywhere adding particular underscores; it overrides 'showstringspaces'
  showstringspaces=false,          % underline spaces within strings only
  showtabs=false,                  % show tabs within strings adding particular underscores
  stepnumber=2,                    % the step between two line-numbers. If it's 1, each line will be numbered
  stringstyle=\color{RubineRed},   % string literal style
  tabsize=2,                       % sets default tabsize to 2 spaces
  title=\lstname,                  % show the filename of files included with \lstinputlisting; also try caption instead of title
  aboveskip=1em,
  belowcaptionskip=0em,
  belowskip=0em
}

\usepackage{hyperref}
\hypersetup{
    colorlinks,
    citecolor=Violet,
    filecolor=black,
    linkcolor=MidnightBlue,
    urlcolor=MidnightBlue
}

\begin{document}
\providecommand{\versionnumber}{0.1.1}
\title{PartitionFinder 1.0 Manual v\versionnumber}
\author{Diego Darriba, David Posada}
\date{November 6, 2015}
\maketitle

\setcounter{tocdepth}{2}
\tableofcontents

\section{Overview}

PartitionFinder is a tool to carry out statistical selection of best-fit partitioning schemes and models of nucleotide substitution / amino acid replacement.

\subsection{Download}

The main project webpage is located at GitHub: \url{http://www.github.com/ddarriba/partitiontest}.

%\subsection{Citation}

%-

\subsection{Disclaimer}

This program is free software; you can redistribute it and/or modify it under the terms of the GNU General Public License as published by the Free Software Foundation; either version 3 of the License, or (at your option) any later version. This program is distributed in the hope that it will be useful, but WITHOUT ANY WARRANTY; without even the implied warranty of MERCHANTABILITY or FITNESS FOR A PARTICULAR PURPOSE. See the GNU General Public License for more details. You should have received a copy of the GNU General Public License along with this program; if not, write to the Free Software Foundation, Inc., 59 Temple Place - Suite 330, Boston, MA 02111-1307, USA. The jModelTest distribution includes Phyml executables.

These programs are protected by their own license and conditions, and using jModelTest implies agreeing with those conditions as well. 

\subsection{Usage}

       Selects the best-fit model of evolution for many-gene DNA or Protein data

       Mandatory arguments to long options are mandatory for short options too

\begin{tabular}{rlp{.5\textwidth}}
       -c& --config-file CONFIG\_FILE &
             Sets the input configuration file. Run with --config-help for more information \\

        &--config-help 
        &      Shows help about configuration files \\

        & --config-template 
        &      Generates a configuration file template \\

       -d& --data-type DATA\_TYPE 
         &     Sets the type of the input data. DATA\_TYPE: nt (nucleotide), aa (amino-acid) \\

         &--disable-ckp 
         &     Disables the checkpointing \\

         &--disable-output 
         &     Disables any file-based output. This option also disables the chackpointing\\

       -F & --empirical-frequencies
          &    Includes models with empirical frequencies (+F) \\
\end{tabular}

       -g, --pergene-bl
              Estimate per-gene branch-lengths

       --force-override
              Existent output files will be overwritten

       -h, --help-
              Displays a help message

       -i, --input-file INPUT\_FILE
              Sets the input alignment file (REQUIRED)

       -k, --keep-branches
              Keep branch lengths from the initial topology. This argument has no effect for initial topology different than fixed

       -N, --non-stop
              Algorithms do not stop if no improvement found at one step, for avoiding local maxima

       -O, --optimize-mode OPTIMIZE\_MODE
              Sets  the  model optimization for the best-fit partition. OPTIMIZE\_MODE: findmodel (find the best-fit model for each partition), gtr (use only GTR model for each partition (nucleic data)
              or AUTO for protein data)
              
       -p, --num-procs NUMBER\_OF\_THREADS
              Number of threads for model evaluation (DEFAULT: 1)

       -r, --replicates NUMBER\_OF\_REPLICATES
              Sets the number of replicates on Hierarchical Clustering and Random search modes

       -s, --selection-criterion CRITERION
              Sets the criterion for model selection. CRITERION: bic (Bayesian Information Criterion) (DEFAULT), aic (Akaike Information Criterion), aicc (Corrected Akaike Information  Criterion),  dt
              (Decision Theory). Sample size for bic, aicc and dt criteria is the alignment length

       -S, --search SEARCH\_ALGORITHM
              Sets  the  search  algorithm  SEARCH\_ALGORITHM:  k1  (evaluate K=1 only), kn (evaluate K=N only), greedy (greedy search algorithm), greedyext (extended greedy search algorithm), hcluster
              (hierarchical clustering algorithm), random (multiple step random sampling), auto (auto-select algorithm (DEFAULT)), exhaustive (exhaustive search)

       -t, --topology STARTING\_TOPOLOGY
              Sets the starting topology for optimization. STARTING\_TOPOLOGY: mp (creates a maximum parsimony topology for each model optimization (DEFAULT)), ml (creates a maximum likelihood topology
              for each model optimization), fixed (uses a fixed ML topology for every model optimization), user (uses a user-defined topology. Requires the "-u" argument). However, if "-u" argument is
              used this option is automatically set

       -T, --get-final-tree
              Conduct final ML tree optimization

       -u, --user-tree TREE\_FILE
              Sets a user-defined topology. This option ignores all starting topologies different from "user-defined". The tree must be in Newick format

       -v, --version
              Output version information and exit
%\end{tabular}

   Exit status:
       0      if OK,

       64     usage error,

       65     error with data,

       69     unavailable feature,

       70     internal software error (probably a bug),

       74     I/O error,

       78     error in configuration.

%\bibliographystyle{natbib}
%\bibliographystyle{achemnat}
%\bibliographystyle{plainnat}
%\bibliographystyle{abbrv}
%\bibliographystyle{bioinformatics}
\bibliographystyle{plain}
\bibliography{biblio}

\end{document}
